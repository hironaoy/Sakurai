\documentclass{jarticle}

\usepackage[deluxe]{otf}
\usepackage{mathrsfs}
\usepackage{amsmath}
\usepackage{amssymb}
\usepackage{framed}
\usepackage[left=20mm, right=20mm, top=20mm, bottom=20mm]{geometry}
\usepackage{physics}
\usepackage{bm}
\usepackage{mathtools}

\mathtoolsset{showonlyrefs=true}

\begin{document}
\noindent
\textbf{J.J. Sakurai \& Jim Napolitano, Modern Quantum Mechanics}
\footnote{
参照せよ:\texttt{github.com/hironaoy/Sakurai}(PDF/\LaTeX)(リポジトリ名を変更しました)
}
\hfill Jun 5, 2024\vspace{-2mm} \\
\hrulefill \\

\noindent
\begin{enumerate}
\item \textgt{系の回転}

  \begin{itemize}
  \item [$\circ$] 回転演算子

    系を回転するに伴って、状態ベクトル$\ket{\alpha}$がどのように回転するかを考える。回転後の系の状態を$\ket{\alpha}$とし、回転前の系の状態を$\ket{\alpha}$としたとき、次のように回転演算子$\mathscr{D}(\varphi, \mathbf{n})$を定義できる。
    \begin{align}
      \ket{\alpha'} = \mathscr{D}(\varphi, \mathbf{n}) \ket{\alpha}
    \end{align}
    回転演算子には次を要請する。
    \begin{align}
      
    \end{align}

  \item [$\circ$] 無限小回転演算子
    
    例によって、無限小の変化を与える演算子から出発する。無限小平行移動演算子、無限小時間発展演算子の形式からの類推に依って、3次元空間のベクトル$\mathbf{n}$方向の軸回りの無限小回転演算子を次で表す。
    \begin{align}
      \mathscr{D}(\mathrm{d}\varphi, \mathbf{n}) = 1 - i\frac{\mathbf{J}\cdot\mathbf{n}}{\hbar}\mathrm{d}\varphi
    \end{align}
    ここで、$\mathbf{J}$を角運動量演算子と定義する。この演算子$\mathbf{J}$にはエルミート性を要請する。

  \item [$\circ$] 有限回転演算子

    有限角度の回転は、無限小回転演算子を無限回作用させることによって実現される。よって、有限回転演算子は次のように構成される。簡単のため$z$軸回りの回転で考えることにする。
    \begin{align}
      \mathscr{D}_z(\varphi) = \lim_{N \rightarrow \infty} \qty[1 - i\frac{J_z}{\hbar} \cdot \frac{\varphi}{N}]^N = \exp(-i\frac{J_z\varphi}{\hbar})
    \end{align}
    系の
  \end{itemize}
\end{enumerate}
\end{document}

